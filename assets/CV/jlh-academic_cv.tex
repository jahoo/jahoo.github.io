%!TEX program = xelatex
\documentclass[11pt,a4paper]{article}

\usepackage{fontspec}
\defaultfontfeatures{Mapping=tex-text,Scale=MatchLowercase}
\setmainfont{Linux Biolinum}
\setsansfont{Linux Biolinum}

%%%%%%%%%%%%%%%%%%%%%%%%%%%%%%%%%%%%%%%%%%%%%%%%%%%%%%%%%%%
%%%%%%%%%%%%%%%% Publications using biblatex-publist %%%%%%
%%%%%%%%%%%%%%%%%%%%%%%%%%%%%%%%%%%%%%%%%%%%%%%%%%%%%%%%%%%
% \newcommand*\publistbasestyle{apa6}%default authoryear
\usepackage[%
natbib=true,
backend=biber,
maxnames=9,
% giveninits=true,
style=publist,
plauthorhandling=highlight,
nameorder=given-family,
plsorting=ddt,
% oainfo=simple,
% marginyear=true,
hlyear=false,
plnumbering=none,%global-descending,
prinfo=false,
linktitles=url,
url=false,%isbn=false,doi=false
]{biblatex}

% Redefine to make year in margin and not in bibentry
% https://github.com/jspitz/biblatex-publist/issues/20
\renewbibmacro*{date}{}
\renewbibmacro*{date+extradate}{}
\renewbibmacro*{issue+date}{}
\renewbibmacro*{bpl:year+labelyear}{\nopunct\ignorespaces}
\newcommand*{\bibyear}{}
\defbibenvironment{bibliography}
{\list%
	{\iffieldequals{year}{\bibyear}
                {}
		{\printfield{year}%
		 \savefield{year}{\bibyear}}}
	{\setlength{\topsep}{0pt}%
	 \setlength{\itemsep}{\bibitemsep}%
	 \leftmargin\labelwidth%
	 \advance\leftmargin\labelsep}}% chktex 41
{\endlist}
{\item}
\renewcommand*\plmarginyear[1]{%
  \raggedleft\normalfont{#1}%
}
% handle multiple variants of name by adding each
\plauthorname[Jacob Louis]{Hoover}
\plauthorname[Jacob]{Hoover}
\plauthorname[Jacob]{Hoover Vigly}
\plauthorname[Jacob Louis]{Hoover Vigly}
\plauthorname[Jacob Louis Hoover]{Vigly}
\plauthorname[Jacob Louis]{Vigly}
\plauthorname[Jacob Louis]{Vigly}
\plauthorname[Jacob]{Vigly}
\addbibresource{/Users/j/Library/texmf/bibtex/bib/all-biblatex.bib}

%%%%%%%%%%%%%%%%%%%%%%%%%%%%%%%%%%%%%%%%%%%%%%%%%%%%%%%%%%%
%%%%%%%%%%%%%%%%%%%%%%%%%%%%%%%%%%%%%%%%%%%%%%%%%%%%%%%%%%%
%%%%%%%%%%%%%%%%%%%%%%%%%%%%%%%%%%%%%%%%%%%%%%%%%%%%%%%%%%%

\usepackage[dvipsnames]{xcolor}
\usepackage[colorlinks]{hyperref}
\usepackage{mathtools} % imports amsmath
\usepackage{cleveref} % must come after importing hyperref and amsmath
\hypersetup{%
  pdftitle={Jacob Louis Hoover Vigly},
  pdfauthor={Jacob Louis Hoover Vigly},
  linkcolor=BrickRed,%
  citecolor=PineGreen,%
  filecolor=Mulberry,%
  urlcolor=Blue!90!Black,%
  menucolor=BrickRed,%
  runcolor=Mulberry,%
  linkbordercolor=BrickRed,%
  citebordercolor=PineGreen,%
  filebordercolor=Mulberry,%
  urlbordercolor=Blue!90!Black,%
  menubordercolor=BrickRed,%
  runbordercolor=Mulberry%
}

% jbib_links: makes nice links, and defines nymdt sorting.
%   Imports mathtools, xcolor, hyperref, cleveref.
% \input{/Users/j/Library/texmf/custom/jbib_links.tex}

% formatting packages
\usepackage[letterpaper, margin=2cm]{geometry}
\usepackage{xunicode,xltxtra,url,parskip,array}
\usepackage{ulem}
%\usepackage{orcidlink}
\normalem%
\RequirePackage{graphicx}
% \usepackage[usenames,dvipsnames]{xcolor}
\usepackage{multicol}
\usepackage{longtable}
% for custom link coloring
% \usepackage{hyperref}
% \definecolor{linkcolour}{rgb}{0,0.4,0.6}
% \hypersetup{colorlinks,breaklinks,urlcolor=linkcolour, linkcolor=linkcolour}

% for \section definition
\usepackage{titlesec}
\titleformat{\section}{\Large\bf\raggedright}{}{0em}{}[\titlerule]
\titlespacing{\section}{0pt}{1pt}{0pt}

\titleformat{\subsection}{\large\bf\raggedright}{}{0em}{}
\titlespacing{\subsection}{0pt}{0pt}{0pt}

%--------------------BEGIN DOCUMENT----------------------
\begin{document}
\pagestyle{empty}
%--------------------TITLE-------------------------------
% \vspace*{-35pt}
% \Huge{Jacob Louis \scshape{Hoover}}
\Huge{Jacob Louis \scshape{Hoover Vigly}}
\small\textit{\hfill{} updated: \today}
%--------------------SECTIONS----------------------------

\begin{tabular}{p{\textwidth-12pt}}
  % \textsc{DOB:}   & 03 Jan 1989 \\
  %\textsc{Address:}   &
  % 1085 av du Dr-Penfield, Montréal, Québec H3A 1A7, Canada\\
  \href{http://jahoo.github.io}{jahoo.github.io}\\
  %\textsc{Phone:}     &
  % +1 508 808 1033\\
  %\textsc{email:}     &
  jacob.hoover [at] mail.mcgill.ca\\
  % \orcidlink{0000-0001-8651-5395}: 0000-0001-8651-5395\\
  %, hooverjac at mila.quebec\\
  %\textsc{website:}   &
\end{tabular}

\vspace*{10pt}

\section{Position}

\begin{longtable}{p{1.7cm}|p{15cm}}
  2024--pres.%
    &Postdoctoral Fellow,
    \textbf{Massachusetts Institute of Technology}, Dept.\ of Brain and Cognitive Sciences\\
    &\href{https://cpl.mit.edu}{Computational Psycholinguistics Lab} | PI: Roger Levy\\
  \end{longtable}

%%%%%%%%%%%%%%%%%%%%%%%%%%%%%%%%%%%%%%
\section{Education}
%%%%%%%%%%%%%%%%%%%%%%%%%%%%%%%%%%%%%%
% \subsection{Degree study}

\begin{longtable}{p{1.7cm}|p{15cm}}
  2018--24%
    &\textsc{Ph.D.},
     \textbf{McGill University},
     Department of \href{https://www.mcgill.ca/linguistics/graduate}{Linguistics}, and
     \href{https://mila.quebec}{\textbf{Mila} -- Québec AI Institute}\\
    &dissertation: ``The Cost of Information: Looking beyond Predictability in Language Processing''\\
    % \citetitle*{hoover.j:2024phd}
    &advisor: Timothy J.\ O'Donnell\\
  %   &[Fast-tracked from MA program in 2019]\\
  \multicolumn{2}{c}{}\\
  % \textsc{Fall '23}%
  %   &Visiting Ph.D.\ student,
  %   \textbf{Massachusetts Institute of Technology}\\
  %   &Department of Brain and Cognitive Sciences, Computational Psycholinguistics Lab\\
  %   &PI: Roger Levy\\
  %   \multicolumn{2}{c}{}\\
  2006--12%
    &\textsc{A.L.B.\ (Bachelor of Liberal Arts)}, \emph{cum laude},
     \textbf{Harvard University \href{https://extension.harvard.edu/}{Extension School}}\\
    &Field of Study: Mathematics | Minor: Linguistics
  \end{longtable}

% \subsection{Non-degree}
%
% \begin{longtable}{p{1.7cm}|p{15cm}}
% \textsc{Summer '19}\hfill& {ESSLLI --- 31st
% European Summer School in Logic, Language and
% Information} | Rīga, Latvia\\
% \textsc{Summer '18}\hfill& {2nd Crete Summer
% School of Linguistics} | Rethymnon, Greece\\
% \textsc{Summer '17}\hfill& {LSA 2017
% Linguistic Institute} | Lexington, KY\\
%%   &courses for grade: Combinatory Categorial
%%   Grammar (Prof.\ Mark Steedman), Computational
%%   Linguistics (Prof.\ Sandra Kübler), Phonology (Prof.\ Adam Albright)\\
% \textsc{Spring '17} \hfill&\textbf{Boston College}
% MATH 4480 \emph{Mathematical Logic} course (Prof.\
% Robert Reed)\\
% \textsc{Summer '16} \hfill& {ESSLLI --- 28th
% European Summer School in Logic, Language and
% Information} | Bolzano, Italy\\
%   & {NY -- St. Petersburg Institute of
%   Linguistics, Cognition and Culture} | St.
%   Petersburg, Russia \\
%%   &short `internship' with Prof.\ Sabine Iatridou
%%   (MIT) on the syntax of coördination.\\
%
% \textsc{Fall '16} \hfill& \textbf{Brown University}
% CLPS 1830, \emph{Grammar and Processing of Ellipsis}
% seminar (Prof. Pauline Jacobson)\\
% \hfill& \textbf{MIT} 24.902, \emph{Syntax} course
% (Prof. David Pesetsky)\\
% \textsc{Fall '13} \hfill&\textbf{University of
% Maryland, College Park}'s \emph{Exploring Quantum
% Physics} | online certificate (Coursera)\\
% \hfill&\textbf{BerkeleyX} CS191x: \emph{Quantum
% Mechanics and Quantum Computation} | online
% certificate (EdX)\\
% \textsc{Spring '03} & \textbf{Framingham State
% University} 61.110, \emph{Languages of the World}
% course with Prof.\ M.\ Mahler \\
% \end{longtable}

  % \subsection{Undergraduate research assistantships}

  % \begin{longtable}{p{1.7cm}|p{15cm}}
  % \textsc{Fall 2013} & Research assistant for Rama
  % Novogradsky (\textbf{BU, Harvard Snedeker Lab})\\
  %       &\footnotesize{Project studying children's early
  %       syntactic development. Helped code and interpret
  %       data.}\\

  % & Research assistant for Maria Polinsky
  % (\textbf{Harvard Polinsky Lab})\\
  %       &\footnotesize{Helped with coding data in an
  %       experiment that looked at the efficacy of
  %       different transcription methods.}\\

  % \textsc{Fall 2009} & Research assistant for Michael
  % Becker and Andrew Nevins (\textbf{Harvard})\\
  %       &\footnotesize{Research project on English
  %       nouns studying voicing assimilation in forming
  %       the plural. Helped in preparing and running
  %       experiments, recruiting participants,
  %       interpreting and writing up data, and
  %       analysis.}\\

  % \textsc{2006--2007} & Research assistant for Michael
  % Becker and Andrew Nevins (\textbf{Harvard})\\
  %       &\footnotesize{Project on voicing assimilation
  %       in Turkish nouns forming the accusative case.
  %       Helped organize contacting/soliciting
  %       participants (Turkish speakers), and gave them
  %       the survey of wugs determining whether they
  %       accepted voicing assimilation.}\\
  % \end{longtable}

  %%%%%%%%%%%%%%%%%%%%%%%%%%%%%%%%%%%%%%
  \section{Teaching}
  %%%%%%%%%%%%%%%%%%%%%%%%%%%%%%%%%%%%%%
  \begin{longtable}{p{1.7cm}|p{15cm}}
    \textsc{Winter '23}%
      & Instructor, McGill COMP/LING 445 \emph{Computational Linguistics}\\
    \textsc{Fall '21}%
      & Guest lecture for McGill COMP/LING 445 \emph{Computational Linguistics}\\
    \textsc{2019--22}%
      & Teaching Assistant, 6 McGill courses in linguistics/computer science\\
      &\quad{}COMP/LING 596 \emph{Probabilistic Programming}
            (\textsc{Winter '21, Winter '22})\\
      &\quad{}COMP/LING 445 \emph{Computational Linguistics}
            (\textsc{Fall '20, '21})\\
      &\quad{}LING 201 \emph{Introduction to Linguistics}
            (\textsc{Fall '19}, \textsc{Winter '20})\\
  \end{longtable}
  % \begin{longtable}{p{1.7cm}|p{15cm}}
  %   Winter 2023&%
  %   Instructor for McGill COMP/LING 445 \emph{Computational Linguistics}\\
  %   \textsc{Winter 2022}&%
  %   TA for McGill COMP/LING 596 \emph{Probabilistic Programming} (Prof.\ Timothy
  %   O'Donnell)\\
  %   % Casual Research Assistant Computational Linguisitcs
  %   \textsc{Fall 2021}&%
  %   Guest lecture for McGill COMP/LING 445 \emph{Computational Linguistics}\\
  %                     & TA for McGill COMP/LING 445 \emph{Computational Linguistics}
  %                     (Prof.\ Timothy O'Donnell)\\
  %   \textsc{Winter 2021}&%
  %   TA for McGill COMP/LING 596 \emph{Probabilistic Programming} (Prof.\ Timothy
  %   O'Donnell)\\
  %   \textsc{Fall 2020}&%
  %   TA for McGill COMP/LING 445 \emph{Computational Linguistics} (Prof.\ Timothy
  %   O'Donnell)\\
  %   \textsc{Winter 2020}&%
  %   TA for McGill LING 201 \emph{Introduction to Linguistics} (Prof.\ Francisco
  %   Torreira, Mathieu Paillé)\\
  %   \textsc{Fall 2019}&%
  %   TA for McGill LING 201 \emph{Introduction to Linguistics} (Prof.\ Francisco
  %   Torreira, Prof.\ Junko Shimoyama)\\
  % \end{longtable}

  %%%%%%%%%%%%%%%%%%%%%%%%%%%%%%%%%%%%%%
  \section{Community}
  %%%%%%%%%%%%%%%%%%%%%%%%%%%%%%%%%%%%%%
  \begin{longtable}{p{1.7cm}|p{15cm}}
    \textsc{2020--22}
    &%
    Graduate student representative, McGill Linguistics Department\\
    2020
    &%
    Graduate Program changes Committee, student member\\
    \textsc{2019--22}
    &%
    Vice President, Academic, \textbf{GLAM}, Graduate Linguistics Association of
    McGill\\
    \textsc{2019--20}&%
    Founding member, \textbf{EARTH-LING}, McGill Linguistics department
    environmental working group\\
    \textsc{2018--}pres.
    &%
    Graduate student member, webmaster, \href{http://mcqll.org}{\textbf{MCQLL}} (Montréal Computational and Quantitative
    Linguistics Lab)\\
  \end{longtable}
  \begin{longtable}[l]{ll}
    Reviewer for:
    &%
    \href{https://transacl.org/}{TACL}, 2022;
    \href{https://cognitivesciencesociety.org/}{CogSci}, 2021;
    \href{https://www.conll.org/}{CoNLL} 2020, 2023
  \end{longtable}



  %%%%%%%%%%%%%%%%%%%%%%%%%%%%%%%%%%%%%%
  \section{Research}
  %%%%%%%%%%%%%%%%%%%%%%%%%%%%%%%%%%%%%%
  % \begin{longtable}{p{1.7}|p{15cm}}
  %   \textsc{2021}&%
  %   \textbf{Linguistic dependencies and statistical dependence}.
  %   \\
  %   \textsc{2020}&%
  %   \textbf{Accounting for variation in number agreement in Icelandic
  %   dative--nominative constructions}.
  %   Proceedings of WCCFL 38, Vancouver, BC. Cascadilla Press. In press.\\
  % \end{longtable}

  % Print bibliography, manual order
  \vspace{5pt}
  \nocite{%
    hoover.j:2024phd,%
    rahimi.h:2024,%
    hoover.j:2023,%
    socolof.m:2022coling,%
    hoover.j:2022amlap,%
    hoover.j:2021emnlp,%
    hoover.j:2021wccfl%
    %hoover.j:2020wccflhandout
  }
  \AtNextBibliography{\small}
  \printbibliography[heading=none]{}
  \vspace{5pt}

  % \subsection{Works in progress}
  % \begin{itemize}
  %   \item \textbf{Predictability and Linguistic Dependencies} (PhD comprehensive
  %     exam project): Using pretrained contextual embedding networks to compare
  %     statistical dependency with syntactic dependencies. Committee: Timothy J,
  %     O'Donnell, Martina Martinović, and Alessandro Sordoni (Microsoft
  %     Research).
  %   \item \textbf{Estonian indefites in fragment answers: from something to
  %     nothing}: Project documenting a semantic puzzle about Estonian indefinites
  %     formed from wh-words, which may be interpreted as either a positive
  %     existential or a negative when uttered elliptically.
  % \end{itemize}

  %%%%%%%%%%%%%%%%%%%%%%%%%%%%%%%%%%%%%%
  \section{Awards}
  %%%%%%%%%%%%%%%%%%%%%%%%%%%%%%%%%%%%%%
  \begin{longtable}{p{1.7cm}|p{15cm}}
    \textsc{2024}&%
      \textbf{National Science Foundation SBE Postdoctoral Research Fellowship}.
      Sponsor: Roger Levy at MIT. ``Algorithmic models of incremental human language
      comprehension.''
       US\$160k.\\
      % 2024-09 through 2026-08
    \textsc{2023}&%
      \textbf{McGill International Graduate Mobility Award},
      for research visit with Roger Levy, MIT\@. CA\$6k.\\
      % 2023-09 through 2023-12
    % \textsc{2020}&%
    %   \textbf{Microsoft Research-Mila Collaboration Grant}, for project
    %   supervised by Alessandro Sordoni (MSR Montreal) and Prof.\ Timothy
    %   O'Donnell (Mila). ``Towards characterizing compositionality and systematic
    %   generalization for natural language representations.'' CA\$27k.\\
    %   % 2020-03 through 2020-12
    \textsc{2019}&%
      \textbf{CRBLM Graduate Scholar Stipend}, Centre for Research on Brain, Language
      and Music (\href{https://crblm.ca/}{CRBLM}). ``What neural network models
      can tell us about linguistic structure.'' CA\$3k.\\
      % Dec 13 2019: amount 3000 CAD
  \end{longtable}

  %%%%%%%%%%%%%%%%%%%%%%%%%%%%%%%%%%%%%%
  \section{Languages}
  %%%%%%%%%%%%%%%%%%%%%%%%%%%%%%%%%%%%%%

  \begin{longtable}[l]{ll}
    English
    &%
      Native\\
    Estonian 
    &%
      Intermediate (2½ years living and working in Estonia:
      \textasciitilde\ CEF level B2)\\
    French 
    &%
      Intermediate (\textasciitilde\ B2)\\
    Japanese
    &%
      Novice (two years' undergraduate study, rather atrophied)\\
    Russian, Spanish%
    &%
      Basic knowledge\\
    % \multicolumn{2}{c}{}\\
    % \textbf{Programming languages}:
    % &%
    %   Clojure, Java, Julia, Python, R
  \end{longtable}

  \section{Ballet career}
  \begin{longtable}{p{1.7cm}|p{15cm}}
    2018--pres.
    &%
    I retired from full-time dancing in 2018, when I began my graduate study at
    McGill. I have continued with occasional appearances as an independent
    artist including with Festival Ballet Providence,  Rhode Island Women's
    Choreography Project,  Revolve Dance Project in Providence, RI,
    and Robinson Ballet in Bangor, ME.\\
    \textsc{2008--18}
    &%
    I worked full-time as professional dancer from 2008 until 2018. Trained as a
    ballet dancer at Walnut Hill School and Miami City Ballet School, I danced
    soloist roles at \textbf{José Mateo Ballet Theatre} in Cambridge, MA
    (2008--12), and \href{http://vanemuine.ee}{\textbf{Vanemuine Theatre}} in
    Tartu, Estonia (2013--15), and was a company dancer for
    \href{http://festivalballetprovidence.org}{\textbf{Festival Ballet
    Providence}}, now Ballet RI (2015--18).\\
  \end{longtable}

  \end{document}
