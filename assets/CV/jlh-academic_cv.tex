%!TEX program = xelatex
\documentclass[11pt,a4paper]{article}

\usepackage{fontspec}
\defaultfontfeatures{Mapping=tex-text,Scale=MatchLowercase}
\setmainfont{Linux Biolinum}
\setsansfont{Linux Biolinum}

% For references
% \newcommand*\publistbasestyle{authoryear}
\usepackage[%
natbib=true,backend=biber,sorting=none,maxnames=9,
style=publist,
plauthorhandling=highlight,
% marginyear,
boldyear=false,
plnumbered=false,
prinfo=false,
linktitleall=true,
url=false,doi=false,isbn=false%
]{biblatex}

% Redefine to make year in margin and not in bibentry
% https://github.com/jspitz/biblatex-publist/issues/20
\renewbibmacro*{date}{}
\renewbibmacro*{date+extradate}{}
\renewbibmacro*{issue+date}{}
\renewbibmacro*{bpl:year+labelyear}{\nopunct\ignorespaces}
\newcommand*{\bibyear}{}
\defbibenvironment{bibliography}
{\list
	{\iffieldequals{year}{\bibyear}
                {}
		{\printfield{year}%
		 \savefield{year}{\bibyear}}}
	{\setlength{\topsep}{0pt}%
	 \setlength{\itemsep}{\bibitemsep}%
	 \leftmargin\labelwidth%
	 \advance\leftmargin\labelsep}}
{\endlist}
{\item}
\renewcommand*\plmarginyear[1]{%
  \raggedleft\normalfont{#1}%
}


\plauthorname[Jacob Louis]{Hoover}
\addbibresource{/Users/j/Library/texmf/bibtex/bib/all-biblatex.bib}
\usepackage[dvipsnames]{xcolor}
\usepackage[colorlinks]{hyperref}
\usepackage{mathtools} % imports amsmath
\usepackage{cleveref} % must come after importing hyperref and amsmath
\hypersetup{%
  linkcolor=BrickRed,%
  citecolor=PineGreen,%
  filecolor=Mulberry,%
  urlcolor=NavyBlue,%
  menucolor=BrickRed,%
  runcolor=Mulberry,%
  linkbordercolor=BrickRed,%
  citebordercolor=PineGreen,%
  filebordercolor=Mulberry,%
  urlbordercolor=NavyBlue,%
  menubordercolor=BrickRed,%
  runbordercolor=Mulberry%
}

% jbib_links: makes nice links, and defines nymdt sorting.
%   Imports mathtools, xcolor, hyperref, cleveref.
% \input{/Users/j/Library/texmf/custom/jbib_links.tex}

% formatting packages
\usepackage[letterpaper, margin=2cm]{geometry}
\usepackage{xunicode,xltxtra,url,parskip,array}
\usepackage{ulem}
%\usepackage{orcidlink}
\normalem
\RequirePackage{graphicx}
% \usepackage[usenames,dvipsnames]{xcolor}
\usepackage{multicol}
\usepackage{longtable}
% for custom link coloring
% \usepackage{hyperref}
% \definecolor{linkcolour}{rgb}{0,0.4,0.6}
% \hypersetup{colorlinks,breaklinks,urlcolor=linkcolour, linkcolor=linkcolour}

% for \section definition
\usepackage{titlesec}
\titleformat{\section}{\Large\bf\raggedright}{}{0em}{}[\titlerule]
\titlespacing{\section}{0pt}{1pt}{0pt}

\titleformat{\subsection}{\large\bf\raggedright}{}{0em}{}
\titlespacing{\subsection}{0pt}{0pt}{0pt}

%--------------------BEGIN DOCUMENT----------------------
\begin{document}
\pagestyle{empty}
%--------------------TITLE-------------------------------
% \vspace*{-35pt}
\Huge{Jacob Louis \scshape{Hoover}}
\small\textit{\hfill updated: \today}
%--------------------SECTIONS----------------------------

\begin{tabular}{p{\textwidth}}
  % \textsc{DOB:}   & 03 Jan 1989 \\
  %\textsc{Address:}   &
  1085 av du Dr-Penfield, Montréal, H3A 1A7, Canada\\
  \href{http://jahoo.github.io}{jahoo.github.io}\\
  %\textsc{Phone:}     &
  % +1 508 808 1033\\
  %\textsc{email:}     &
  jacob.hoover [at] mail.mcgill.ca\\
  % \orcidlink{0000-0001-8651-5395}: 0000-0001-8651-5395\\
  %, hooverjac at mila.quebec\\
  %\textsc{website:}   &
\end{tabular}

\vspace*{10pt}

%%%%%%%%%%%%%%%%%%%%%%%%%%%%%%%%%%%%%%
\section{Education}
%%%%%%%%%%%%%%%%%%%%%%%%%%%%%%%%%%%%%%
% \subsection{Degree study}

\begin{longtable}{p{1.7cm}|p{15cm}}
  2018--(2024)%
    &%
    PhD student in Linguistics,
    \textbf{\href{https://www.mcgill.ca/linguistics/graduate}{McGill} University}, and
    \href{http://mila.quebec}{\textbf{Mila}} -- Quebec Artificial Intelligence
    Institute\\
    &%
    advisor: Timothy J.\ O'Donnell. [Fast-tracked from MA program in 2019]\\
    \multicolumn{2}{c}{}\\
  2006--2012%
    &%
    \textsc{Bachelor of Liberal Arts (ALB)}, \emph{cum laude},
    \textbf{Harvard University \href{https://extension.harvard.edu/}{Extension School}}\\
    &%
    Field of Study: Mathematics | Minor: Linguistics\\
    %|\textsc{GPA}: 3.92 \\
    %&  Awards:\\
    %& - Reginald H. Phelps Prize (awarded to top three in
    %graduating class) \\
    %& - Dean's list Academic Achievement Award (each
    %year) \\
  \end{longtable}

  % \subsection{Summer schools}

  % \begin{longtable}{p{1.7cm}|p{15cm}}
  % \textsc{Summer 2019}\hfill& \textbf{ESSLLI --- 31st
  % European Summer School in Logic, Language and
  % Information} | Rīga, Latvia\\
  % \textsc{Summer 2018}\hfill& \textbf{2nd Crete Summer
  % School of Linguistics} | Rethymnon, Greece\\
  % \textsc{Summer 2017}\hfill& \textbf{LSA 2017
  % Linguistic Institute} | Lexington, KY\\
  %   &courses for grade: Combinatory Categorial
  %   Grammar (Prof.\ Mark Steedman), Computational
  %   Linguistics (Prof.\ Sandra Kübler), Phonology (Prof.\ Adam Albright)\\
  % \textsc{Summer 2016} \hfill& \textbf{ESSLLI --- 28th
  % European Summer School in Logic, Language and
  % Information} | Bolzano, Italy\\
  %   & \textbf{NY -- St. Petersburg Institute of
  %   Linguistics, Cognition and Culture} | St.
  %   Petersburg, Russia \\
  %   &short `internship' with Prof.\ Sabine Iatridou
  %   (MIT) on the syntax of coördination.\\
  % \end{longtable}

  % \subsection{Courses outside of program}

  % \begin{longtable}{p{1.7cm}|p{15cm}}
  % \textsc{Spring 2017} \hfill&\textbf{Boston College}
  % MATH 4480 \emph{Mathematical Logic} course (Prof.\
  % Robert Reed)\\
  % \textsc{Fall 2016} \hfill& \textbf{Brown University}
  % CLPS 1830, \emph{Grammar and Processing of Ellipsis}
  % seminar (Prof. Pauline Jacobson)\\
  % \hfill& \textbf{MIT} 24.902, \emph{Syntax} course
  % (Prof. David Pesetsky)\\
  % \textsc{Fall 2013} \hfill&\textbf{University of
  % Maryland, College Park}'s \emph{Exploring Quantum
  % Physics} | online certificate (Coursera)\\
  % \hfill&\textbf{BerkeleyX} CS191x: \emph{Quantum
  % Mechanics and Quantum Computation} | online
  % certificate (EdX)\\
  % % \textsc{Spring 2003} & \textbf{Framingham State
  % University} 61.110, \emph{Languages of the World}
  % course with Prof.\ M.\ Mahler \\
  % \end{longtable}

  % \subsection{Undergraduate research assistantships}

  % \begin{longtable}{p{1.7cm}|p{15cm}}
  % \textsc{Fall 2013} & Research assistant for Rama
  % Novogradsky (\textbf{BU, Harvard Snedeker Lab})\\
  %       &\footnotesize{Project studying children's early
  %       syntactic development. Helped code and interpret
  %       data.}\\

  % & Research assistant for Maria Polinsky
  % (\textbf{Harvard Polinsky Lab})\\
  %       &\footnotesize{Helped with coding data in an
  %       experiment that looked at the efficacy of
  %       different transcription methods.}\\

  % \textsc{Fall 2009} & Research assistant for Michael
  % Becker and Andrew Nevins (\textbf{Harvard})\\
  %       &\footnotesize{Research project on English
  %       nouns studying voicing assimilation in forming
  %       the plural. Helped in preparing and running
  %       experiments, recruiting participants,
  %       interpreting and writing up data, and
  %       analysis.}\\

  % \textsc{2006--2007} & Research assistant for Michael
  % Becker and Andrew Nevins (\textbf{Harvard})\\
  %       &\footnotesize{Project on voicing assimilation
  %       in Turkish nouns forming the accusative case.
  %       Helped organize contacting/soliciting
  %       participants (Turkish speakers), and gave them
  %       the survey of wugs determining whether they
  %       accepted voicing assimilation.}\\
  % \end{longtable}

  %%%%%%%%%%%%%%%%%%%%%%%%%%%%%%%%%%%%%%
  \section{Teaching}
  %%%%%%%%%%%%%%%%%%%%%%%%%%%%%%%%%%%%%%
  \begin{longtable}{p{1.7cm}|p{15cm}}
    Winter 2023&%
    Instructor for McGill COMP/LING 445 \emph{Computational Linguistics}\\
    \textsc{Winter 2022}&%
    TA for McGill COMP/LING 596 \emph{Probabilistic Programming} (Prof.\ Timothy
    O'Donnell)\\
    % Casual Research Assistant Computational Linguisitcs
    \textsc{Fall 2021}&%
    Guest lecture for McGill COMP/LING 445 \emph{Computational Linguistics}\\
                      & TA for McGill COMP/LING 445 \emph{Computational Linguistics}
                      (Prof.\ Timothy O'Donnell)\\
    \textsc{Winter 2021}&%
    TA for McGill COMP/LING 596 \emph{Probabilistic Programming} (Prof.\ Timothy
    O'Donnell)\\
    \textsc{Fall 2020}&%
    TA for McGill COMP/LING 445 \emph{Computational Linguistics} (Prof.\ Timothy
    O'Donnell)\\
    \textsc{Winter 2020}&%
    TA for McGill LING 201 \emph{Introduction to Linguistics} (Prof.\ Francisco
    Torreira, Mathieu Paillé)\\
    \textsc{Fall 2019}&%
    TA for McGill LING 201 \emph{Introduction to Linguistics} (Prof.\ Francisco
    Torreira, Prof\. Junko Shimoyama)\\
  \end{longtable}

  %%%%%%%%%%%%%%%%%%%%%%%%%%%%%%%%%%%%%%
  \section{Community}
  %%%%%%%%%%%%%%%%%%%%%%%%%%%%%%%%%%%%%%
  \begin{longtable}{p{1.7cm}|p{15cm}}
    2020
    &%
    Graduate Program changes Committee, student member\\
    \textsc{2020--2022}
    &%
    Graduate student representative, McGill Linguistics Department\\
    \textsc{2019--2022}
    &%
    Vice President, Academic, \textbf{GLAM}, Graduate Linguistics Association of
    McGill\\
    \textsc{2019--2020}&%
    Founding member, \textbf{EARTH-LING}, McGill Linguistics department
    environmental working group\\
    \textsc{2018--}pres.
    &%
    Graduate student member, webmaster, Montreal Computational and Quantitative
    Linguistics Lab (\href{http://mcqll.org}{\textbf{MCQLL}})\\
  \end{longtable}
  Reviewing: TACL 2022, CogSci 2021, CoNLL 2020

  %%%%%%%%%%%%%%%%%%%%%%%%%%%%%%%%%%%%%%
  \section{Awards}
  %%%%%%%%%%%%%%%%%%%%%%%%%%%%%%%%%%%%%%
  \begin{longtable}{p{1.7cm}|p{15cm}}
    \textsc{2020}&%
      \textbf{Microsoft Research-Mila Collaboration Grant}, for project
      supervised by Alessandro Sordoni (MSR Montreal) and Prof.\ Timothy
      O'Donnell (Mila)\\
      % Towards characterizing compositionality and
      % systematic generalization for natural language
      % representations
    \textsc{2019}&%
      \textbf{Graduate Scholar Stipend}, Centre for Research on Brain, Language
      and Music (\href{https://crblm.ca/}{CRBLM})\\
      % Dec 13 2019: amount 3000 CAD
  \end{longtable}

  %%%%%%%%%%%%%%%%%%%%%%%%%%%%%%%%%%%%%%
  \section{Research}
  %%%%%%%%%%%%%%%%%%%%%%%%%%%%%%%%%%%%%%
  % \begin{longtable}{p{1.7}|p{15cm}}
  %   \textsc{2021}&%
  %   \textbf{Linguistic dependencies and statistical dependence}.
  %   \\
  %   \textsc{2020}&%
  %   \textbf{Accounting for variation in number agreement in Icelandic
  %   dative--nominative constructions}.
  %   Proceedings of WCCFL 38, Vancouver, BC. Cascadilla Press. In press.\\
  % \end{longtable}

  % Print bibliography, manual order
  \vspace{5pt}
  \nocite{%
    hoover.j:2022psyarxiv%
    ,hoover.j:2022amlap%
    ,hoover.j:2021emnlp%
    ,hoover.j:2021wccfl%
    %,hoover.j:2020wccflhandout
  }
  \AtNextBibliography{\small}
  \printbibliography[heading=none]{}
  \vspace{5pt}

  % \subsection{Works in progress}
  % \begin{itemize}
  %   \item \textbf{Predictability and Linguistic Dependencies} (PhD comprehensive
  %     exam project): Using pretrained contextual embedding networks to compare
  %     statistical dependency with syntactic dependencies. Committee: Timothy J,
  %     O'Donnell, Martina Martinović, and Alessandro Sordoni (Microsoft
  %     Research).
  %   \item \textbf{Estonian indefites in fragment answers: from something to
  %     nothing}: Project documenting a semantic puzzle about Estonian indefinites
  %     formed from wh-words, which may be interpreted as either a positive
  %     existential or a negative when uttered elliptically.
  % \end{itemize}

  %%%%%%%%%%%%%%%%%%%%%%%%%%%%%%%%%%%%%%
  \section{Languages}
  %%%%%%%%%%%%%%%%%%%%%%%%%%%%%%%%%%%%%%

  \begin{longtable}[l]{ll}
    English
    &%
      Native\\
    Estonian 
    &%
      Intermediate (2½ years living and working in Estonia:
      \textasciitilde\ CEF level B2)\\
    French 
    &%
      Intermediate (\textasciitilde\ B1)\\
    Japanese
    &%
      Novice (two years' university study, at this point rather atrophied)\\
    Russian, Spanish%
    &%
      Basic knowledge\\
    \multicolumn{2}{c}{}\\
    \textbf{Programming languages}:
    &%
      Clojure, Java, Julia, Python, R
  \end{longtable}

  \section{Ballet career}
  \begin{longtable}{p{1.7cm}|p{15cm}}
    2018--pres.
    &%
      I retired from full-time dancing in 2018, when I began my graduate study
      at McGill, but I have continued with occasional appearances as a
      freelancer and as a guest artist including with Festival Ballet Providence, 
      RI, and Robinson Ballet in Bangor, ME.\\
    \textsc{2008--2018}
    &%
      Trained as a ballet dancer at Walnut Hill School and Miami City Ballet
      School. I worked full-time as professional dancer from 2008 until 2018. I
      danced soloist roles when at \textbf{José Mateo Ballet Theatre} in
      Cambridge, MA, (2008--2012) and principal roles when at the
      \href{http://vanemuine.ee}{\textbf{Vanemuine Theatre}} in Tartu, Estonia
      (2013--2015), and was a company dancer for
      \href{http://festivalballetprovidence.org}{\textbf{Festival Ballet}} in
      Providence, RI (2015--2018).\\
  \end{longtable}

  \end{document}
